\hyphenation{No-SQL}
\hyphenation{da-ta-ba-ses}
\hyphenation{ser-vers}
\hyphenation{pa-ra-mount}

\section{Introduction}
\label{sec:intro}

% Scale, scale, scale
Modern Internet applications employ data stores that scale to the entire population 
of online users. For example, personalized content recommendation services  require 
maintaining profiles for hundreds of millions of unique users. Traditional SQL data 
management systems cannot scale up with these requirements, leading to a new generation 
of not-only-SQL, or NoSQL, databases -- e.g., Google's Bigtable~\cite{BigTable2006}, Apache Hadoop HBase\footnote{\small{\url{http://hbase.apache.org/}}}, etc. These technologies have been designed 
for extreme simplicity (key-value store API), scalability (data partitioning across thousands of machines), 
and reliability (redundant storage). In parallel, the proliferation of affordable high-end hardware 
(multi-core CPU's, inexpensive RAM, SSD storage) enabled building NoSQL databases capable of serving 
data at memory speeds~\cite{Corfu2012,RAMCloud2011,FDPlus2012}.

% Transactions are great
Historically, NoSQL databases only allowed atomic reads and writes of individual items. 
More recent systems (e.g., Google's Percolator~\cite{Percolator2010} and HBase's 
Omid\footnote{\small{\url{https://github.com/yahoo/omid}}}~\cite{Omid2014})
introduce transaction processing~\cite{GrayTP1993} for complex applications that
require ACID semantics while accessing multiple items. NoSQL transaction processors
implement consistency models extensively studied by the database community~\cite{Papadimitriou1979, FeketeTODS2005}.
They have been shown to scale well with the database size. 

% We have a problem
Transaction processing does not come for free. Every transaction incurs latency penalties at its begin 
and commit boundaries. In throughput-oriented applications that perform long
transactions latency is not of big concern, and indeed the overhead is minor~\cite{Omid2014}. 
However, it is well-pronounced in online, interactive settings, which are
the main focus of this work.
The faster the underlying database is, the larger the toll. Table~\ref{tab:latencies} exemplifies the impact of ``transactifying'' HBase 
reads and writes by Omid in a high-speed environment (fully evaluated in
Section~\ref{sec:eval}).
Latencies start growing as every operation is framed as a transaction (column
2).
They double as part of the traffic is batched in short and long transactions
(columns 3 and 4, respectively).

%\setlength{\belowcaptionskip}{0pt}
\setlength{\abovecaptionskip}{3pt}
\begin{table} [t]
\centering{
\begin{tabular}{ccccc}
%& {\bf \small{Native}} & {\bf \small{Transactified}} & {\bf
%\small{Transactified} }    & {\bf \small{Transactified} } \\
& {\bf native} & {\bf trans} & {\bf
trans }    & {\bf trans } \\
%&                    &                              & {\bf short transactions}
% & {\bf medium-size transactions} \\
&                    &        {\bf + none }         & {\bf + short
} & {\bf + long } \\
%&                    &                              & {\bf in the background} &
% {\bf in the background} \\
\hline
{\bf Read}  & 3.9 & 5.2 & 6.0 & 9.2 \\
{\bf Write}  & 8.3 & 9.5 & 10.3 & 14.0  \\
\end{tabular}
}
\caption{\bf{\small{The impact of transactification
(\emph{trans}) on HBase \emph{native} operations latencies (ms).
Transaction processing adds a surplus that grows with the length of transactions
executed in the background (\emph{none}, \emph{short}, \emph{long}).
}}}
\label{tab:latencies}
\end{table}

% Why this is nontrivial
We would like to avoid automatically converting the atomic database operations
into transactions. Unfortunately, running them side by side with transactional traffic on shared data 
without any coordination is error-prone. Consider, for example, an imaginary social application, 
in which user statuses can be either directly posted by the users, or speculated by the system, 
based on a variety of signals (status history, location, time of day, sensor data, friends' posts,  etc.). 
In the latter case, the system updates the status unless the user has recently
posted a new one.
Therefore, it performs a transaction that (1) reads the user's status, and possibly some other data, 
(2) does some computation, and (3) writes the status back. In contrast, a human-originated status 
update is blind -- it must complete in the real time, and needs not be transactional. In a %na\"{\i}ve 
non-coordinated implementation, the transaction is not aware of this update,
and may overwrite it with a stale speculated value.
%, so the update is \emph{lost}.

An additional problem with the non-coordinated design is exposure to uncommitted data.
The transaction processing layer prevents transactions from reading each other
intermediate modifications that may eventually roll-back~\cite{GrayTP1993}.
However, in a heterogeneous environment, native reads can retrieve transaction's
\emph{dirty} writes.

% Goal and challenge
Our goal is to preserve the original performance of native operations while
maintaining the familiar consistency guarantees for them as well as for
transactions. This challenge is amplified in distributed databases, in which the data is partitioned among 
multiple servers. In this context, any solution must take care not to impede
the datapath scalability, by introducing minimum synchronization.     

We present {\em Mediator} --- Mixed Database Access Transaction Oracle --- 
a scalable transaction processor that %allows to benefit from both worlds. 
guarantees data consistency in the presence of native operations. To the best 
of our knowledge, this problem has not been addressed by the database community before. 
%The contributions of our work are as follows.

%Theoretical model
\remove{We establish a formal definition of the consistency model for systems
supporting transactions and native operations, and prove Mediator's compliance
with it.}
We establish a consistency model for systems
supporting transactions and native operations. Namely, we extend the popular
{\em serializability\/}~\cite{Papadimitriou1979} and {\em snapshot isolation\/} (SI)~\cite{FeketeTODS2005} models to accommodate 
the native traffic semantics. The extension is not straightforward since native operations 
are not captured as transactions, and the consistency requirements are relaxed for them. 
The formal definition of the models and Mediator's correctness proofs are
deferred to the full version of this paper~\cite{full}.

%Distributed Algorithm
%We present an algorithm that satisfies the required safety properties.
Similarly to earlier work~\cite{WeikumTIS2001,Omid2014}, 
Mediator exploits multi-version concurrency control at the database layer 
to implement its consistency model. The unique challenge is installing 
a logical order between transactions, which are ordered by a centralized logical 
clock, and native operations, accessing multiple independent 
servers. We introduce \emph{temporal fencing\/} -- a novel protocol that
loosely synchronizes the servers' local clocks with the global clock. 
The algorithm trades performance optimization of native operations 
for an extra overhead imposed on transactions. 

We implement a working prototype of Mediator on top of HBase, and extensively evaluate it 
on a distributed testbed. We study Mediator's performance tradeoffs by comparing
it to an Omid-powered system that automatically converts native operations into transactions. The results 
emphasize the performance impact incurred to native traffic by Omid.
More importantly, we show that Mediator's {\em overall\/} system performance is superior 
for a vast majority of the considered mixed traffic patterns. In particular, its throughput 
is higher for all workloads that contain at least $50\%$ native operations. 
%Therefore, a slight overhead incurred to the transactional traffic is justified 
%by preserving the native traffic's performance. 
%The bulkier the transactional traffic is, the more advantageous our system becomes. 

The rest of this paper is structured as follows. Section~\ref{sec:overview} sketches Mediator's system architecture. 
Section~\ref{sec:semantics}  
informally presents mixed traffic semantics, and 
Section~\ref{sec:algorithm} describes the algorithms implementing two
consistency models.
Section~\ref{sec:eval} depicts and analyzes the evaluation results. 
%Rigorous model definition and correctness proofs appear in
% Section~\ref{sec:correctness}.
Finally, we survey related work in Section~\ref{sec:related}, 
and conclude with Section~\ref{sec:conc}.
