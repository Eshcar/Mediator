\hyphenation{su-pports}

\section{Model and Correctness}
\label{sec:correctness}

\remove{
A \emph{database} is a collection of \emph{data items}  uniquely identified by
their keys.
%(in our context, table rows). 
Multiple versions (values) of each item may co-exist 
simultaneously. Each item is a single-writer object, controlled by a 
unique process called {\em data server}. 
%The latter installs  total order among the item's versions. %A data server can atomically write multiple items
%under its control. 

A \emph{transactional processing system} employs \emph{transactions} to execute
pieces of sequential code, namely \emph{operations}, by concurrent asynchrounous
processes. An {implementation} of the system provides routines to execute these
operations. When a process calls a routine we say it \emph{invokes} an
operation, when  the execution of the routine is completed a \emph{response} is
returned.
%A \emph{transaction} is a sequence of operations executed by a single process.
% These include gets and puts on data items, as well as begin, commit and abort operations.
A \emph{begin} operation initializes a transaction and returns 
its handle. A \emph{commit} operation returns an indication whether the transaction 
committed or aborted. An \emph{abort} cancels the transaction and returns an
abort indication.
A \emph{get} operation specifies the key of the data item to read, and returns the value read
by the operation. A \emph{put} operation specifies the key and a value to be
written. Get and put operations return an abort indication if the transaction
in which it is invoked has to abort.
For simplicity of presentation, we do not consider additional operations (e.g., range get queries).
In the rest of the paper we use the terms put/write, and get/read
interchangeably.

After the transaction begins, it executes a sequence of get and put operations, followed by either a commit or an abort. 
\emph{Abort ratio} %of an execution
is the ratio between aborted transactions and transactions that have not invoked the abort operation. 

The collection of data items accessed by a transaction is its data set. 
The data items it puts are its \emph{write set}, and the data items it gets are its 
\emph{read set}. A \emph{read-only} transaction performs only get 
operations (its write set is empty); otherwise, it is an \emph{update}
transaction. A \emph{write-only} transaction is an update transaction with an empty read
set. A \emph{put} transaction is a write-only transaction updating a single
item.

A \emph{native} operation is either a put or a get operation that is not
executed in the context of a transaction. Unlike their transactional
equivalents, native operation never return an abort indication.

%An \emph{implementation} of a transaction processing system provides (1) data
% structures for representing transactions and data items, and (2) algorithms, specified as 
%operations on these data structures. \emph{Asynchronous} processes 
%invoke these operations in order to execute native gets/puts and transactions.
}

We refine the classical definitions of serializability
%~\cite{Papadimitriou1979}
and snapshot isolation (SI) %~\cite{Berenson1995, WeikumTIS2001} 
to apply for mixed traffic executions. Then we outline the safety proofs for our
algorithms.


%We assume native operations are never aborted by the database, and require that
% the transaction processing implementation never aborts them as well.

\subsection{Consistency Models}
\label{sec:models}


A \emph{transactional processing system} employs \emph{transactions} to execute
pieces of sequential code, namely \emph{operations}, by concurrent asynchrounous
processes. An \emph{implementation} of the system provides routines to execute
these operations. When a process calls a routine we say it \emph{invokes} an
operation, when  the execution of the routine is completed a \emph{response} is
returned.

An \emph{execution} is a sequence of invocation and responses of native and
transactional operations starting from the initial configuration of the database. 
%Each operation is assumed to be executed as an atomic step. 
If the last response of a transaction is a commit or abort
indication, then the transaction is \emph{completed}, and is said to be
\emph{committed} or \emph{aborted}, respectively; if a transaction invoked a
commit but not yet received a response then it is \emph{commit-pending};
otherwise, it is \emph{active}. 

In a \emph{serial execution}, native operations and transactions are executed to
completion in isolation one after the other.
\remove{A \emph{committed projection} is attained by 
%keeping all native operations and committed transactions, while 
discarding all active and aborted transactions from the execution.}

A transaction $T$ is \emph{legal} in a % committed projection of a 
serial execution if every read invocation of $x$ in $T$ returns a value that was
written to $x$ in $T$ before the read, or if there are no invocations of write to $x$ in $T$
before the read, then the read returns a value that was last written to $x$
before $T$ by a native put or a committed transaction, or the initial value of
$x$ if no such write to $x$ exists before $T$. A native read $op$ is
\emph{legal} in a % committed projection of a 
serial execution if it returns a value that was last written to $x$
before $op$ by a native put or a committed transaction, or the initial value of
$x$ if no such write to $x$ exists before $op$. A % committed projection of a
serial execution is \emph{legal} if all its committed transactions and native
reads are legal.

%Two executions are \emph{equivalent} if (1)~they consist of the same set of
% transactions and native operations, (2)~both transactional and native gets return the same values in both executions, and 
%(3)~the order of any pair of transactions executed by the same process, as well
%as any pair of read and write operations (native and transactional) accessing
%the same item by the same process is the same in both executions
% (\emph{per-process order}).
An execution $\alpha'$ preserves the \emph{per-process transaction order} of
execution $\alpha$ if it preserves the order of any pair of begin operations executed
by the same process in $\alpha$. An execution $\alpha'$ preserves the
\emph{per-process item order} of execution $\alpha$ if it preserves the order of
any pair of native and transactional operations accessing the
same item and executed by the same process in $\alpha$.

Let $T$ be a committed transaction in an execution $\alpha$. Let $T|read$
($T|write$, $T|all$) be the longest subsequence of $T$ in
$\alpha$ consisting only of $T$'s read (write, read and write, respectively)
invocations and their corresponding responses. 
The transaction $T_r$ is defined as follows: if $T|read$ is empty
then $T_r$ is empty, otherwise $T_r$ is the sequence of a begin invocation and
response appended by $T|read$, a commit invocation and a committed response;
$T_w$ and $T_{rw}$ are defined in a similar way, w.r.t. $T|write$ and
$T|all$.

A \emph{serialization} of an execution $\alpha$ is the sequence
$\sigma_{\alpha}$ that includes a \emph{serialization point} for every native
operation in $\alpha$, $*_{op}$, for every committed transaction, and for
some of the commit-pending transactions in $\alpha$,
$*_{T}$.
The \emph{serial execution} $\alpha'$ corresponding to
$\sigma_{\alpha}$ is defined by replacing each $*_{T}$ with $T_{rw}$ and each
$*_{op}$ with \emph{op}'s invocation and response in
$\alpha$. 


\begin{definition}[serializability] 
An implementation supports \emph{serializability} if any execution $\alpha$ 
has a
serialization, $\sigma_{\alpha}$, such that
the serial execution
$\alpha'$ corresponding to $\sigma_{\alpha}$ is legal and preserves
the per-process transaction order and per-process item order of $\alpha$.
\end{definition}
%$S$ is called the \emph{serialization} of $E$.
%; we assume that this serialization order preserves the per-process order, i.e., transactions of the same process maintain their order.

\remove{
The weaker condition of snapshot isolation 
%was suggested as an efficient alternative to serializability for relational databases. It 
decouples the consistency of the gets and the puts of a transaction.
%into the begin time of the transaction and its commit time, respectively. 
Informally, all reads within a transaction see a consistent view of the
database, as though the transaction operates on a private snapshot of the
database taken before its first read. In addition, concurrent transactions are not allowed to modify the same data.  
%this means each transaction is provided with a snapshot of the database at the time of transaction start. 
\full{For a formal definition, see~\cite[Definition 10.3]{WeikumTIS2001}.}
}
\remove{
A \emph{read point} of a transaction is a sequence of its begin operation
followed by all its get operations. A \emph{write point} of a transaction is a
sequence of its put operations followed by its commit operation.  
In a \emph{snapshot serial execution}, the read points of all transactions, the
write points of all transactions, and all native operations are executed to
completion in isolation one after the other, where each read point precedes its
respective write point.   
The \emph{interval} of a write-only transaction is its commit operation. The
interval of other update transactions starts with the begin operation and ends
with the commit operation of this transaction (if no such commit operation
exists, the interval ends at the end of the execution). 
The interval of a native operation is the operation iteslf.
Two transactions or a transaction and a native operation \emph{overlap} if their
intervals overlap. 
%; otherwise they are non-overlapping.

\begin{definition}[snapshot isolation]
An implementation supports \emph{snapshot isolation} if it satisfies two properties:

\textbf{Snapshot property} A committed projection of any execution is equivalent to some snapshot-serial execution. 
%$S$, such that $S$ satisfies two properties: 

%\noindent
\textbf{Disjoint writes property} The write sets of any pair of overlapping transactions, and any pair of overlapping 
transaction and native put
operation, are disjoint.

%\noindent
%\textbf{Snapshot property} A get, either transactional or native, reading item $r$, returns the last value written to $r$ prior to the transaction's reading point or the native read's, respectively, or the initial value if no such write exists\footnote{At any point in snapshot-serial executions, the last value written to an item is well defined.}.
\end{definition}
}


A \emph{snapshot serialization} of an execution $\alpha$, is the sequence
$\sigma_{\alpha}$ that includes a serialization point for every native
operation in $\alpha$, $*_{op}$, and for every committed transaction and for
some of the commit-pending transactions in $\alpha$ it includes a \emph{read
serialization point} $*_{T_r}$ and a \emph{write serialization point} $*_{T_w}$
such that (i)~$*_{T_r}$ preceeds $*_{T_w}$, (ii)~both $*_{T_r}$ and $*_{T_w}$
are inserted after the invocation of $T$'s begin and before the response of
$T$'s commit in $\alpha$.
The \emph{snapshot serial execution} $\alpha'$ corresponding to
$\sigma_{\alpha}$ is defined by replacing each $*_{T_r}$ with $T_r$ and each
$*_{T_w}$ with $T_w$  and each $*_{op}$ with
\emph{op}'s invocation and response in $\alpha$.

For any execution $\alpha$, consider a snapshot serialization
$\sigma_{\alpha}$ of $\alpha$. The \emph{interval} of a native put
operation \emph{op} is $*_{op}$. The \emph{interval} of a
write-only transactions $T$ is the point $*_{T_w}$. The \emph{interval} of a
read-only transactions $T$ is the point $*_{T_r}$. The \emph{interval} of a
read-write transaction $T$ is the interval between
$*_{T_r}$ and $*_{T_w}$ in $\sigma_{\alpha}$. 
Two transactions or a transaction and a native
put operation \emph{overlap} in $\sigma_{\alpha}$ if their intervals overlap.

\remove{
The \emph{interval} of a write-only transaction is the interval between the
invocation and response of its commit operation.
The \emph{interval} of an update transaction (that is not write-only) is the
interval between the invocation of its begin operation and the response of its commit
operation. If no such response exists, the interval ends at the end of
the execution).
The \emph{interval} of a native operation is the interval between its
invocation and response. Native operations are \emph{atomic}, namely within
their intervals there is no invocation or response of any other operation.
Two transactions or a transaction and a native operation \emph{overlap} if their intervals overlap.
}


\begin{definition}[snapshot isolation]
\remove{
An execution $\alpha$ is \emph{snapshot serializable}, if it has a snapshot
serialization $\sigma_{\alpha}$ such that the serial execution $\alpha'$
corresponding to $\sigma_{\alpha}$ is legal and preserves the per-process item
order in $\alpha$.
}

An implementation supports \emph{snapshot isolation} if any execution $\alpha$ 
%satisfies two properties:
has a snapshot
serialization $\sigma_{\alpha}$ such that:
%\noindent

\textbf{Snapshot property} the snapshot serial execution $\alpha'$
corresponding to $\sigma_{\alpha}$ is legal and preserves the per-process item
order in $\alpha$.
%$\alpha$ is snapshot serializable. 

%\noindent
\textbf{Disjoint writes property} The write sets of any pair of overlapping
transactions, and any pair of overlapping transaction and native put operation
in $\sigma_{\alpha}$, are disjoint.  
\end{definition}

The snapshot property implies that native get operations cannot read an
uncommitted value; for serializability, the same trivially follows from the total order.
The deviation of these definitions from a straightforward extension of classical 
definitions --conceiving native operations as transactions---is twofold:
(1)~native operations cannot abort (while transactions can), 
(2)~similar to traditional NoSQL, both models might flip the order of two
native gets in the serialization (whereas two read transactions by the same
process cannot flip their order). 
For example, a process that sequentially reads two items might observe an older version in the 
second read (e.g., since the items are 
updated in parallel by some transaction).
%In our social application example, when the application displays
%a view covering the status of all the user's friends, it uses native reads
%to access them, and it is ok if one status is not
%up-to-date (as it is being concurrently updated by a transaction).
