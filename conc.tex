\section{Conclusions}
\label{sec:conc}

We presented Mediator -- a transaction processing system
for Web-scale NoSQL databases. Mediator mitigates the consistency 
gaps that arise when transactional and native operations 
are allowed to share the same data in a straightforward way. Mediator 
protects the safety invariants of both API's -- namely, (1) atomic reads 
and writes for the native traffic, and (2) snapshot isolation or serializability
for the transactional traffic. 

Mediator provides weak synchronization between two types of logical clocks: 
the global clock maintained by the transaction processing service, and the 
local clocks of multiple independent database servers. This temporal
fencing mechanism installs a logical order between native
and transactional accesses, despite the fact that the native accesses completely 
bypass Mediator's infrastructure. The protocol is well-founded, and also 
extremely lightweight compared to physical clock synchronization.

Mediator preserves the original performance of native traffic, while incurring minor impact on transactional operations. A large-scale evaluation 
shows that this design choice strikes a favorable tradeoff. Namely, it demonstrates
that performance-wise, Mediator's approach is superior to automatic transactification 
of native operations, for a vast majority of our tested workloads. We also
show that spurious aborts -- the price paid for preserving 
the best of both worlds -- are very infrequent. 
%which introduces a considerable overhead, 



