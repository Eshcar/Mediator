\section{Motivating Example}
\label{sec:example}

Consider a mobile-commerce recommender system where users {\em check-in\/} 
in the new locations they arrive to, and before long receive a personalized
recommendation of local businesses or deals. The suggestion is based 
on a collaborative filtering model~\cite{KorenAdvancesCF2011} learned from the 
shopping data -- i.e., similar items are recommended to users with similar traits. 

Figure~\ref{fig:example} depicts the system's implementation. For each item, the {\sc Items} table 
holds its mathematical model and location. The {\sc Users} 
table holds, for each user, the model, current location, and the recommendation. 
The database is updated both directly by the mobile users that consume the 
recommendations and update the locations, and by a backend process that continuously 
updates the stale recommendations. The recommendation is invalidated (erased) upon 
the user's check-in in a new location.

The backend process consists of two parts -- model update and suggestion computation. 
The former consumes a stream of $\langle user, item, rating\rangle$ interaction events,
and updates the user and item models for each event. 
The latter scans the users with empty recommendations, and for 
each user (1) ranks the co-located items using the user and item models, and 
(2) updates the recommendation with the top results.

Both backend jobs access multiple tables for read and write. In order to update the  
models or to compute the recommendations in a consistent way, they need to employ 
transactions. However, the check-in is a ``blind-write'' that has no reason to be 
transactified. On the contrary, it is part of an interactive flow, and hence, 
any transaction-processing overhead is undesirable for it. However, serving it with 
the native database API is unsafe. Since the transaction processing infrastructure 
is unaware of native updates, some transaction might simply run over one. 
For example, a recommendation invalidated by a new check-in might be overwritten by the 
values corresponding to the user's previous location, hence the user might not receive 
a fresh suggestion on time. In this context, the expected behavior of the 
backend pipeline would be to detect the conflict, roll its transaction back, 
and re-compute the result. 

\begin{figure}
\includegraphics[width=3.3in]{Figs/example.epsi}
\vskip .1in
\caption{\bf{\small{Motivating example -- an m-commerce system that computes local 
business recommendations for mobile users. The system combines online native data 
accesses with offline flows that use transactions. }}}
\label{fig:example}
%\vskip -.2in
\end{figure}
