\hyphenation{acr-oss}

\section{Related Work}
\label{sec:related}

Transaction processing is a textbook area in database research~\cite{WeikumTIS2001,GrayTP1993}.
It appears in the ANSI SQL standards, 
%\footnote{\footnotesize{\url{http://www.contrib.andrew.cmu.edu/~shadow/sql/sql1992.txt}}}, 
as well as in modern NoSQL technologies that took databases to an unprecedented
scale (e.g.,~\cite{Percolator2010}). The literature defines a wealth of transaction 
consistency models that capture different perceptions of concurrency control 
(e.g.,~\cite{Papadimitriou1979, Berenson1995}). Traditionally, client applications 
sharing a single database instance agree on a single consistency model (or multiple 
levels thereof that subsume each other), and pay the required performance toll. 
%Google's recent paper database~\cite{Spanner2012} claims: {\em we believe it is better 
%to have application programmers deal with performance problems due to overuse of transactions 
%as bottlenecks arise, rather than always coding around the lack of transactions.} 
We posit that this approach is not necessarily required, i.e., it is possible 
to accommodate within the same database two incompatible semantics: 
multi-operation transactional consistency, and atomic read-write consistency
appropriate for simple key-value store applications. 

The database community has been reasoning about transaction semantics since the late 
70's~\cite{Papadimitriou1979}. Serializability %model, which is the most 
%intuitive way to logically serialize a concurrent execution, 
has been widely adopted.
%Two-phase locking is a widely accepted implementation of serializability~\cite{GrayTP1993}. 
The seminal paper by Berenson et al.~\cite{Berenson1995} %discussed the
% advantages and drawbacks of serializability and its different relaxations, and 
introduced the snapshot isolation model. The latter is particularly attractive due to its 
implementations that improve concurrency. 
%
\remove{Fekete et al. laid foundations for making SI executions
serializable~\cite{FeketeTODS2005}.
and authored a series of papers comparing the semantics and performance 
of serializability and snapshot isolation.
%showed that the semantic gap between the two is small, and 
A host of later work~\cite{Cahill2008,Ports2012} suggested practical ways of realizing 
these ideas. Mediator implements both SI and serializability semantics. Its implementation 
is {\em optimistic} -- it lets multiple transactions proceed without interference until commit, 
upon which data conflicts are identified, and some transactions might be aborted~\cite{GrayTP1993}. 
}
\remove{
While that paper focused on restructuring the application code to avoid anomalies, Cahill 
et al.~\cite{Cahill2008} suggested a different approach. Their concurrency control algorithm, 
Serializable Snapshot Isolation, ensures that every execution is serializable, no matter what the 
program does. PostgreSQL was the first database product to adopt this algorithm~\cite{Ports2012}.
}

%Snapshot isolation achieves better overall throughput than serializability, 
%at the expense of aborting some transactions~\cite{Alomari2008, Cahill2008}. 
%We show similar results.  

In databases that support range queries, the literature distinguishes between {\em repeatable
read\/} (RR) and {\em serializability\/} isolation levels, %The former
% guarantees serial execution.The latter 
which subsumes RR, and extends it with a requirement of avoiding {\em phantom
reads\/} (returning two different tuple sets for the same key range to two queries running under the same 
transaction~\cite{GrayTP1993}).  Should Mediator be extended to support
predicate queries, it can use the same transaformation technique
by Fekete et al.~\cite{FeketeTODS2005} to get a phantom-free serializable
implementation.
% phantom freedom through range-sensitive conflict resolution,
%however this is not essential for its core algorithm.
  
Early NoSQL databases, e.g., Google Bigtable~\cite{BigTable2006}, Yahoo! 
PNUTS~\cite{PNUTS2008} and Apache HBase sacrifice strong consistency for extreme 
scalability. Their safety guarantee is single-key atomic reads and writes~\cite{AttiyaDC2004}. 
Google Percolator technology~\cite{Percolator2010} supports multi-operation transactions in Bigtable for incremental maintenance 
of its search index. Percolator
%, which is embedded in Bigtable, 
implements snapshot isolation through database locks. Omid, a transaction 
processor for HBase~\cite{Omid2014}, supports snapshot isolation with a
lock-free protocol. Omid also implements serializability~\cite{OmidEurosys2012}.
%its implementation is external to the database. 
Mediator's design bears similarity with Omid,
however, its algorithm is profoundly different, to capture the new safety properties. 

Google Spanner~\cite{Spanner2012} provides distributed transactions 
across datacenter with a blend of SI (for read-only transactions) and serializability guarantees. 
Spanner implements lock-free read-only transactions and lock-based read-write
transactions. 
%It coordinates locks through a synchronous protocol that can
%tolerate limited discrepancies among physical clocks across multiple locations.
Calvin~\cite{calvin} also addresses globally distributed transactions, albeit in
a different way. It replaces locking with deterministic scheduling that orders
transactions through a global consensus service. 
%Similarly to other transaction processing systems,
SCORe~\cite{Peluso2012} is a serializable partial replication protocol that
guarantees read operations always access a consistent snapshot. It applies a
timestamp management scheme to synchronize the nodes handling the
transaction. Neither one of the above
%Spanner nor Calvin
provide any consistency guarantees to hybrid workloads targeted by Mediator. 
%Investigating how to adapt 
%Mediator to multi-datacenter environments is beyond the scope of our work.

\remove{In a way, Mediator challenges the generic claim in~\cite{Spanner2012}:
``We believe it is better to have application programmers deal with performance problems 
due to overuse of transactions as bottlenecks arise, rather than always coding around 
the lack of transactions.''}

%TM
\emph{Transactional memory} (TM)~\cite{Herlihy2008} is a popular approach for alleviating the difficulty of programming concurrent applications for multi-core and multiprocessing systems. 
TM allows concurrent processes synchronize via in-memory transactions.
\remove{
In practice, TM must allow accessing the same items from inside and outside a transaction; this is crucial both for interoperability with legacy code and for improving the performance of the TM. \emph{Privatization}
mechanisms~\cite{Herlihy2008} allow the programmer to ``isolate'' some items making them private to a process. 
The process can thereafter access them non-transactionally, without interfering with other processes.
}
%RingSTM
Our TSO implementation is inspired by %Spear et al.'s work on 
RingSTM~\cite{Spear2008} -- an %the first TM 
implementation that allows accessing the same items from inside and outside a transaction.
%is inherently livelock-free, while at the same time permitting parallel write-back by concurrent disjoint transactions. 
%In RingSTM, transactions commit by enqueuing an entry onto a global ring,
%and represent their read and write sets by Bloom filters. 
RingSTM is not geared for distributed environments, 
hence our challenges are different.
%thus reducing the overhead of metadata manipulation and inspection. 




